\chapter{Compilation}

In the previous two chapters, we have shown the source language that we wish
to express our programs in, and the target language that models the machine
code that is actually used by the real computers to perform computations.

In a way, those two languages are the complete opposites of each other:
the first one is about composing functions, allows no side effects such
as assignment and provides an implicit memory model. The other makes
memory operations explicit, allows to exchange information solely with the
use of assignment, and the only thinkable way of performing composition
is by sequencing operations and subprograms.

The transformation from the first sort of languages to the second
has traditionally been called \textit{compilation}, and some of its
popular techniques will be presented in this chapter.

We shall begin with transforming Scheme programs into a special form
that does not contain any nested function calls, and hence should be
easier to transform to the program on our machine.
 
\section{Continuation-Passing Style}

Consider the following procedure for computing the coefficient
$\Delta = b^2 - 4ac$ that is helpful in finding the roots of a quadratic
equation $ax^2+bx+c=0$:

\begin{Snippet}
(define (delta a b c)
  (- (* b b) (* 4 (* a c))))
\end{Snippet}

Prior to computing the value of the whole expression, we need
to have our machine compute the values of sub-expression and
store them somewhere.

Provided that our machine has a sufficient number of registers,
we could expect it to compile to the following sequence of machine
instructions\footnote{Note that we use a new register to hold the
  result of each intermediate computation. Of course, the real machines
  usually have a limited number of registers, but the assumption
  that each register is assigned exactly once leads to the form
  of programs called \textit{Static Single-Assignment form} (or
  SSA for short), which is used, for example, in the machine language
  of the LLVM virtual machine.
}:

\begin{Snippet}
  (bb <- b * b)
  (ac <- a * c)
  (4ac <- 4 * ac)
  (bb-4ac <- bb - 4ac)
\end{Snippet}

Let us now ask the opposite question: given a sequence of machine
instructions, how can we express them in Scheme (or $\lambda$ calculus)?

We typically imagine that a von Neumann machine operates by altering
its current state (and indeed, this is how we implemented our virtual
machine in Scheme).

However, we could imagine that there is something quite different
going on: each assignment to a register can be perceived not as
actually altering some value, but as creating a new scope where
the original variable has been shadowed with a new one (bound with
the altered value), and where \textit{the rest of the program}
is evaluated.

In order to clarify things a bit, we can define an auxiliary
function \texttt{pass} that takes a value and a procedure and
simply passes the value to the procedure:
\begin{Snippet}
  (define (pass value procedure)
    (procedure value))
\end{Snippet}

This way, we could rewrite programs like

\begin{Snippet}
  (bb <- b * b)
  ... the rest of the program ...
\end{Snippet}

as

\begin{Snippet}
  (pass (* b b)
    (lambda (bb)
      ... the rest of the computation ...))
\end{Snippet}

The procedure that represents the rest of the computation has
traditionally been called a \textit{continuation}, and the form
of a program where control is passed explicitly to continuations
is called \textit{continuation-passing style}\cite{SussmanSteele1976}.

If we were to define our \texttt{delta} procedure using the
continuation-passing style, we would need to extend its
argument list with a continuation, i.e. a parameter that
would explain what to do next with the value that our function
has computed. Moreover, we could demand that all the functions
that we use behave in the same way, i.e. that instead of the
function \texttt{pass}, we would have functions \texttt{pass*}
and \texttt{pass-} that would compute the values of operations
\texttt{*} and \texttt{-} and pass them to their continuations.
This way we make sure that there are no nested expressions
in our program.

\begin{Snippet}
  (define (pass-delta a b c continuation)
    (pass* b b
      (lambda (bb)
        (pass* a c
          (lambda (ac)
            (pass* 4 ac
              (lambda (4ac)
                (pass- bb 4ac
                  (lambda (bb-4ac)
                    (continuation bb-4ac))))))))))
\end{Snippet}

It should be rather clear that this form of representing programs
isn't particularly handy, and should only be used as an intermediate
or transient representation of a program.

Of course, the above program is rather straightforward, as it consists
only of a sequence of applications of primitive functions. To make
the correspondence between continuation-passing style and normal
Scheme programs more complete, we need to consider general function
applications and conditional expressions.

Let us consider the program that computes an \textit{absolute value}
of a number. In Scheme, we would define it as

\begin{Snippet}
(define (abs n)
  (if (is n >= 0)
    n
    (- n)))
\end{Snippet}

The corresponding machine code would look more or less like this:

\begin{Snippet}
  (if n >= 0 goto 2)
  (n <- 0 - n)
  ;; the rest of the program
\end{Snippet}

which in turn loosely corresponds to the following continuation-passing
function

\begin{Snippet}
(define (pass-abs n continuation)
  (if (>= n 0)
    (continuation n)
    (pass- 0 n continuation)))
\end{Snippet}

The above code was trivial in that it used the condition that is directly
representable in our machine code. However in general we can place arbitrarily
nested Scheme expressions as the \texttt{if}'s \texttt{<condition>} clauses.

Continuations can also be used for returning multiple values. For example,
the code finds the roots of a quadratic equations
would need to check whether the discriminant $\Delta$ is non-negative
in order to proceed with the computation of the roots:

\begin{Snippet}
  (define (quadratic-roots a b c)
    (cond ((is (delta a b c) > 0)
           (values (/ (- (- b) (sqrt (delta a b c))) (* 2 a))
                   (/ (+ (- b) (sqrt (delta a b c))) (* 2 a))))
          ((is (delta a b c) = 0)
           (/ (- b) (* 2 a)))))
\end{Snippet}

Note that we have used the \texttt{values} form that is used in Scheme
for returning multiple values. Although we didn't introduce it to be the
part of our host language, its meaning in the context of the discussion
regarding continuation-passing style is obvious (we simply pass more than
one value to the continuation). Of course, we could have instead returned
a list of values, which in turn would force us to fix on some particular
representation of lists, which we want to avoid at this moment.

Note also, that we defined \texttt{quadratic-roots} to return meaningful
values only if the \texttt{delta} is either \texttt{zero?}
or \texttt{positive?} -- that is, if it is negative, then the expression
\texttt{quadratic-roots} has no values (or in other words, its value
is \textit{unspecified}).

Lastly, some readers may find it displeasing, that we didn't capture
the value of \texttt{(delta a b c)} using the \texttt{let} form
(which is something that we would normally do to avoid some redundant
computations, and more specifically, not to repeat ourselves).
We ask those readers to be forgiving, as our goal at this point
is to demonstrate the correspondence between various Scheme programs
and their CPS counterparts, rather than promote good programming
practices. Or in other words, we are in the position of a surgeon
performing an operation on a patient. Of course, it is in general
good for health to jog, but it would be insane to recommend
jogging to someone who is lying on an operation table with open
veins.

The computation of \texttt{quadratic-roots} is a bit tricky

\begin{Snippet}
  (define (pass-quadratic-roots a b c continuation)
    (pass-delta a b c
      (lambda (delta#1)
        (if (> delta#1 0)
          (pass-delta a b c
            (lambda (delta#2)
              (pass-sqrt delta#2
                (lambda (sqrt@delta#3)
                  (... (continuation -b-sqrt@delta/2a#6
                                     -b+sqrt@delta/2a#11)
                        ...)))))
        ;else
          (pass-delta a b c
            (lambda (delta#12)
              (if (= delta#12 0)
                (pass- 0 b
                  (lambda (-b#13)
                    (pass* 2 a
                      (lambda (2a#14)
                        (pass/ -b#16 2a#14
                          (lambda (-b/2a#15)
                            (continuation -b/2a#15)))))))
              ;else
                (continuation))))))))
\end{Snippet}

Some (rather trivial) parts of the code for computing roots were
omitted for clarity. It should be clear now that for complex conditions
we simply compute the value of a condition, and then pass it to
a continuation that takes the result and, depending on its value,
either executes the CPS version of its \texttt{<then>} branch or
the CPS version of its \texttt{<else>} branch.

Note also, that -- in order to avoid accidental name clashes --
we generated a new name for the result of each evaluated (or executed)
expression.

Let's now consider the following definition of the \texttt{factorial}
function:
\begin{Snippet}
  (define (factorial n)
    (if (= n 0)
        1
    ;else
        (* n (factorial (- n 1)))))
\end{Snippet}

We can imagine that its continuation-passing version could
look like this:

\begin{Snippet}
  (define (pass-factorial n continuation)
    (if (= n 0)
      (continuation 1)
    ;else
      (pass- n 1
        (lambda (n-1)
          (pass-factorial n-1
            (lambda (n-1!)
              (pass* n n-1!
                (lambda (n*n-1!)
                  (continuation n*n-1!)))))))))
\end{Snippet}

The questions that we need to ask are: (1) how do we transform
arbitrary functional Scheme code to continuation passing style
and (2) how do we transform continuation passing style program
to machine code.

\section{Conversion to Continuation-Passing Style}

The meta-circular evaluator presented in chapter 2 performed
case analysis on the shape of expression to be evaluated,
and had to consider six cases: \texttt{lambda} form, \texttt{quote}
form, \texttt{if} form, function application, symbols and
numbers. It did not deal with the \texttt{define} form, as
it could be expressed using \texttt{lambda} and fixed point
combinators.

For the purpose of conversion to continuation-passing style,
we shall consider the \texttt{define} forms as well, because
although they shouldn't be strictly necessary, the implementation
of recursion in machine code is rather straightforward (certainly
more so than of fixed point combinators).

The core of the transformation is the \texttt{passing} function,
which takes an expression and produces its continuation-passing
counterpart.

It takes an additional argument, a continuation expression, to
which it shall pass the value of the transformed expression.

In case of quoted values, it only invokes the continuation:

\begin{Snippet}
(define (passing expression continuation)
  (match expression
    (('quote _)
     `(,continuation ,expression))
\end{Snippet}

The case of conditional expression is a bit trickier:
we need to pass the value of a \textit{condition} to a new continuation,
which -- depending on that value -- passes either the value
of the \textit{then} branch, or the value of the \textit{else}
branch to the original continuation. Note that we need to
provide an original name for the result of the condition, to avoid
accidental name clashes:

\begin{Snippet}
    (('if <condition> <then> <else>)
     (let ((result (original-name <condition>)))
       (passing <condition>
         `(lambda (,result)
	    (if ,result
	        ,(passing <then> continuation)
	        ,(passing <else> continuation))))))
\end{Snippet}

The continuation-passing version of the \texttt{lambda} form should receive
additional argument -- a continuation -- and the body should be converted
to the continuation-passing style\footnote{Although in this work we have
  been consequently passing the continuation as the last argument for the
  purpose of clarity, in practice it might be a better idea to make it
  the first argument, because that would allow to handle variadic functions
  properly.}. Since procedures are first-class values,
they shall be passed to the original continuation just like quoted values.

\begin{Snippet}
    (('lambda <args> <body>)
     `(,continuation (lambda (,@<args> return)
                       ,(passing <body> 'return))))
\end{Snippet}

Function application is the trickiest bit: we need to compute the
values of all the compound arguments, passing them to the subsequent
continuations, which finally invoke the continuation-passing version
of the called function (obtained using the \texttt{passing-function}
function), passing its result to the original continuation\footnote{Note
  that, for clarity of presentation, we depart from the definition
  of Scheme in that we do not allow complex expression in the head
  (function) position.}

\begin{Snippet}
    ((function . arguments)
     (let ((simple-arguments (map (lambda (argument)
				    (if (compound? argument)
					(original-name argument)
                                    ;else
                                        argument))
				  arguments)))
       (passing-arguments arguments simple-arguments
			  `(,(passing-function function)
			    ,@simple-arguments
			    ,continuation))))
\end{Snippet}

Otherwise, the expression is just a value to be passed to the continuation:

\begin{Snippet}
    (_
     `(,continuation ,expression))))
  ;; the definition of ``passing'' ends here
\end{Snippet}

The \texttt{passing-arguments} helper function (used for dealing with
applications) takes three arguments: a list of arguments to the
called function, a list of new names for the compound arguments,
and a final call to be made from the nested chain of continuations:

\begin{Snippet}
(define (passing-arguments arguments names final)
  (match arguments
    (()
     final)

    ((argument . next)
     (let (((name . names) names))
       (if (compound? argument)
	   (passing-arguments next names
			      (passing argument
                                `(lambda (,name) ,final)))
       ;else
	   (passing-arguments next names final))))))
\end{Snippet}

The \texttt{passing-program} function takes a program, that is,
a sequence of definitions and an expression, and converts each of the
definitions to the continuation-passing style. For the sake
of simplicity, we shall assume here, that all the definitions
are function definitions. Note that we need to pass the additional
\texttt{return} argument that is stripped away after the conversion.

We assume, that the program passes its result to the \texttt{exit}
continuation.

\begin{Snippet}
(define (passing-program program)
  (let (((('define names functions) ... expression) program))
    `(,@(map (lambda (name function)
               (let ((('return pass-function) (passing function
                                                       'return)))
	          `(define ,(passing-function name)
                           ,pass-function)))
	     names functions)
      ,(passing expression 'exit))))
\end{Snippet}

The complete code, with the implementations of \texttt{passing-function}
and \texttt{original-name} can be found in the appendix \ref{compiler}.

We can check, that the value of

\begin{Snippet}
(passing-program '((define !
                     (lambda (n)
	               (if (= n 0)
                           1
			   (* n (! (- n 1))))))
		    (! 5)))
\end{Snippet}

is the form

\begin{Snippet}
((define pass-!
   (lambda (n return)
     (pass= n 0
       (lambda (n=0/1)
         (if n=0/1
            (return 1)
         ;else
            (pass- n 1
	      (lambda (n-1/3)
                (pass-! n-1/3
		  (lambda (!/n-1/2)
	            (pass* n !/n-1/2 return))))))))))
 (pass-! 5 exit))
\end{Snippet}

\section{Generating machine code}

Usually, function serves as an \textit{abstraction barrier} in
complex systems: we sometimes imagine it as a black box that
for some given input it produces some output. As such, functions
are often perceived as \textit{compilation units}: a single function
corresponds to a distinguished block of compiled code, along with its
entry point.

When designing an interface for the abstraction barrier on an actual
system, we need to answer the following questions:
\begin{itemize}
\item How are the parameters passed on to a function?
\item How are the values returned from a function?
\end{itemize}

On our machine, the possible answers are that arguments and
values can either be passed through registers, through stack
or through the memory heap.

Typically, passing values through registers is most efficient
and therefore most desirable. However, since the number of
registers in a CPU is usually small, some other conventions
often need to be established (for example, the first few
arguments can be passed through registers, and another ones
through the stack or heap).

Another question is, how can a function know where the
control should be transferred after it finishes its execution.
Typically, this information is stored on the \textit{call stack},
which stores the appropriate address in the caller code.

However, while the use of stack is in general inevitable,
sometimes it may be more desirable to store the return address
in a register, and only save it when invoking another function
(because this can decrease the number of memory accesses, which
are typically more expensive than register access).

The latter option, although may seem less obvious, allows to
perceive function calls as gotos that pass arguments\cite{SussmanSteele1976},
where the return address is just another argument to be passed.

The advantage of this approach is that if a call to another
function is the last thing that a calling function does, then
the called function can simply inherit the return address
from the caller. This trick is called \textit{Tail Call
  Optimization} and allows, among other things, to 
express loops using recursion.

Let's recall our continuation-passing style version of the factorial
function:

\begin{Snippet}
(define pass-!
  (lambda (n return)
    (pass= n 0
      (lambda (n=0/1)
        (if n=0/1
           (return 1)
        ;else
           (pass- n 1
             (lambda (n-1/3)
               (pass-! n-1/3
                 (lambda (!/n-1/2)
                   (pass* n !/n-1/2 return/1))))))))))
\end{Snippet}

The function is recursive, but it is not tail recursive,
because it calls another continuation from within the recursive
call.

We expect it to be transformed to something similar to the following
assembly code:

\begin{Snippet}
  factorial:
    (if n <> 0 goto else:)
    (result <- 1)
    (goto return)
  else:
    (n-1 <- n - 1)
    (push n)
    (push return)
    (n <- n-1)
    (return <- proceed:)
    (goto factorial:)
  proceed:
    (pop return)
    (pop n)
    (n-1! <- result)
    (n*n-1! <- n * n-1!)
    (result <- n*n-1!)
    (goto return)
\end{Snippet}

In addition to registers corresponding to the continuation arguments
(i.e. \texttt{n}, \texttt{n-1}, \texttt{n-1!} and \texttt{n*n-1!}),
there are two additional registers -- \texttt{return}, which stores
the return address of current procedure, and \texttt{result}, which
is used for passing the function's result to the caller.

One can see that, prior to the recursive call, we had to store
the return address on the stack and then restore it after the return
from the call. We also had to save and restore the value of the
\texttt{n} register, because it was used as an argument to the
\texttt{factorial:} procedure, but its original value was used
in the sequel of the procedure.

Let's now consider the tail-recursive version of the procedure,
which takes an additional argument -- the accumulator -- to store
the result:

\begin{Snippet}
(define !+
  (lambda (n a)
    (if (= n 0)
        a
    ;else
        (!+ (- n 1) (* n a)))))
\end{Snippet}

Its continuation-passing version looks like this

\begin{Snippet}
(define pass-!+
   (lambda (n a return)
     (pass= n 0
       (lambda (n=0/1)
         (if n=0/1
           (return a)
         ;else  
           (pass* n a
             (lambda (n*a/3)
               (pass- n 1
                 (lambda (n-1/2)
                   (pass-!+ n-1/2 n*a/3 return))))))))))
\end{Snippet}

which in turn roughly corresponds to the following assembly code:

\begin{Snippet}
  factorial+:
    (if n <> 0 goto else:)
    (result <- a)
    (goto return)
  else:
    (n*a <- n * a)
    (n-1 <- n - 1)
    (n <- n-1)
    (a <- n*a)
    (goto factorial+:)
\end{Snippet}

The code does not perform any stack operations, and it is clear
that the function call is performed just as a simple goto with
register assignment.

Note that the calling function has to know the names of the registers
that are used to pass arguments to the called function. It may also
have to know what registers are used by the called function internally
(including the registers used by all functions that are called by
the called function, as well as registers used by the functions
called by these functions, and so on) in order to know whether
it should save their values on the stack before the call, and
restore them afterwards.

Also, the code generated by our procedure is wasteful with regard
to the number of used registers. Normally, computers have a limited
number of registers, and compilers try to reuse them as much as possible
in order to minimize the number of accesses to RAM (which is typically
much slower than manipulating register values).

It would therefore be more realistic to rename the arguments to functions
in a systematic way, and also minimize the number of registers that
are used within a procedure (this process is called \textit{register
allocation} in the literature\cite{WikipediaRegisterAllocation} \cite{Keep2013}).

However, since these issues have very little to do with the merit of
this work, we will proceed with our assumption, that the number of
registers of our machine is sufficient to perform any computation
we desire (which, at this very moment, is either computing a factorial
or -- ultimately -- sorting an array).

We therefore assume that each calling function knows at least the
names of registers for each defined procedure, that will be available
via \texttt{argument\--names} helper function.

\begin{Snippet}
(define (passing-program->assembly program/CPS)
  (let (((('define names passing-functions)
	  ...
	  expression) program/CPS))

    (define (argument-names function-name)
      (any (lambda (name ('lambda (arguments ... return) . _))
             (and (equal? name function-name)
                  arguments))
           names passing-functions))
\end{Snippet}

The main procedure, invoked recursively for each defined function,
as well as for the main expression of the program, should be able
to transform an expression in CPS form to a piece of assembly.

There are actually only three cases that we need to consider:
branching, invocation of a function, and returning a value. Since
-- as we mentioned earlier -- upon invocation, the program
may need to save the information contained in the registers
whose content might be overwritten by a called function
-- we need to track the registers used up to a given point.

The code for returning a value to the continuation is rather
trivial -- we assign the desired value to the \texttt{result}
register and perform a jump to the address contained in the
\texttt{return} register:

\begin{Snippet}
    (define (assembly expression/cps registers)
      (match expression/cps
        (('return value)
         `((result <- ,value)
           (goto return)))
\end{Snippet}

The code for branching is a bit tricky, as we need
to undo some of the effects of our CPS transformation, to handle
the conditionals properly (as we noted in chapter 2, Scheme
provides Boolean values \texttt{\#true} and \texttt{\#false}, but
here -- for simplicity -- we assume, that the instruction
\texttt{(if a >?< b goto c)} can only be generated from the
code of the form \texttt{(pass<?> a b (lambda (a<?>b) (if a<?>b ...)))}.
Furthermore, to attain some readability, we shall inverse the
condition in the comparison, and perform jump to the \texttt{else}
branch). We use the \texttt{sign} function, which converts names
like \texttt{pass=} or \texttt{pass+} to operators like \texttt{=}
or \texttt{+}.

We generate a new label for the \texttt{<else>} branch,
add \texttt{a} and \texttt{b} to the set of used registers,
generate a branching instruction followed by assembly
for the \texttt{<then>} expression, followed by the label
for the \texttt{<else>} branch, followed by machine code
for the \texttt{<else>} expression.

\begin{Snippet}
        ((pass<?> a b ('lambda (a<?>b) ('if a<?>b
                                            <then>
                                            <else>)))
          (let ((else (new-label 'else))
                (registers (union registers
                                  (maybe-register a)
                                  (maybe-register b))))
            `((if ,a ,(inversion (sign pass<?>)) ,b goto ,else)
              ,@(assembly <then> registers)
              ,else
              ,@(assembly <else> registers))))
\end{Snippet}

For clarity, invocation should be handled by a separate
function:

\begin{Snippet}
        ((operator . operands)
         (call operator operands registers))))
    ;; the definition of ``assembly'' ends here
\end{Snippet}

We need to take the following factors into consideration:
\begin{itemize}
\item whether the continuation is a \texttt{lambda} expression
  or a \texttt{return} expression
\item whether the current operator is primitive (like \texttt{pass+})
  or a defined function (like \texttt{pass-factorial} from our
  example), or a \texttt{lambda} expression (anonymous function)
\end{itemize}

\begin{Snippet}
    (define (call operator operands registers)
      (cond ((primitive-operator? operator)
             (call-primitive operator operands registers))

            ((defined-function? operator)
             (call-defined operator operands registers))

            ((anonymous-function? operator)
             (call-anonymous operator operands registers))))
\end{Snippet}

``Calling'' a primitive operator is easy -- we simply transform
it to assembly instruction, followed by the code generated from
the body of the continuation (or a \texttt{(goto return)}
instruction in the case of a call to the \texttt{return}
continuation)

\begin{Snippet}
    (define (call-primitive operator operands registers)
      (let (((left right continuation) operands))
        (match continuation
          (('lambda (result) body)
           `((,result <- ,left ,(sign operator) ,right)
             ,@(assembly body (union registers
                                     (maybe-register left)
                                     (maybe-register right)
                                     `(,result)))))
          (_ ;; a ``return'' continuation
           `((result <- ,left ,(sign operator) ,right)
             (goto ,continuation))))))
\end{Snippet}

In order to call a defined procedure, we need to save the
arguments that we might be using in the future. At this stage,
we could perform a fairly elaborate analysis to find out
which registers that are used by our function after the call
are overwritten by the called functions, and only save those.

However -- again, for the sake of simplicity -- we shall only
check, which registers are going to be needed after we return
from the call (and we only do so if there's actually anything
to be done after the return -- otherwise we should perform the
tail call optimization).

We save registers by performing a series of \texttt{push}
instruction, and restore them by executing \texttt{pop}
in the reverse order. 

\begin{Snippet}
    (define (call-defined function arguments registers)
\end{Snippet}
\begin{Snippet}
      (define (save registers)
        (map (lambda (register)
               `(push ,register))
             registers))
\end{Snippet}
\begin{Snippet}
      (define (restore registers)
        (map (lambda (register)
               `(pop ,register))
             (reverse registers)))
\end{Snippet}
\begin{Snippet}
      (define (pass values function)
        (let ((names (argument-names function)))
          (map (lambda (name value)
                 `(,name <- ,value))
               names values)))
\end{Snippet}
\begin{Snippet}
      ;; the body of ``call-defined'' begins here
      (let (((arguments ... continuation) arguments)
            (entry (passing-function-label function)))
\end{Snippet}
\begin{Snippet}
        (match continuation
          (('lambda (result) body)
           (let* ((proceed (new-label 'proceed))
                  (sequel (assembly body registers))
                  (registers (intersection registers
                                  (used-registers sequel))))
             `(,@(save registers)
               ,@(pass arguments function)
	       (push return)
	       (return <- ,proceed)
               (goto ,entry)
               ,proceed
	       (pop return)
               ,@(restore registers)
               ,@sequel)))
\end{Snippet}
\begin{Snippet}
          (_ ;; tail call optimization
           `(,@(pass arguments function)
	     (goto ,(passing-function-label function)))))))
\end{Snippet}

Given all these helper functions, we can now express the
compilation of a whole program\footnote{
  A careful reader probably noticed that we're lacking
  the definitions of \texttt{anonymous-function?} and
  \texttt{call-anonymous}. These definitions are trivial,
  as the call to anonymous function boils down to register
  assignment followed by execution of assembly code of the
  body of that function. They would contribute nothing
  to the examples presented here, so we allowed ourselves
  to omit them here. They are of course available in the
  appendix \ref{compiler}.
}. As noted earlier, we assume
that a program is a sequence of function definitions followed
by a single expression. We therefore need to compile both the
definitions and the expression. Furthermore, we need to
take into account what should happen after our program finishes
its execution. Obviously, we want our machine to \texttt{halt}.

\begin{Snippet}
    ;; body of ``passing-program->assembly'' begins here
    `((return <- end:)
      ,@(assembly expression '())
      ,@(append-map (lambda (name ('lambda args body))
		      `(,(passing-function-label name)
			,@(assembly body '())))
		    names passing-functions)
      end:
      (halt))))
    ;; the definition of ``passing-program->assembly'' ends here
\end{Snippet}

We can observe that the programs for computing factorial function
behave roughly as we expected them to: the tail recursive version
does not perform any stack operations and only uses \texttt{goto}
to transfer control. The other version saves the \texttt{return}
register on the stack prior to the call, along with other registers
that are needed in the sequel.

\section{Conclusion}

Although the compiler presented in this chapter successfully transforms
some high level functions to efficient machine code, it is of course
by no means complete. It does not handle higher order functions properly,
nor does it support arbitrary precision arithmetic. Moreover,
it does not perform any register allocation and uses a new register
for storing each intermediate result, which makes it inapplicable
to real machines. It does, however, serve its purpose, in that it
gives a rough overview of the compilation process.

